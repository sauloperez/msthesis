%% This is an example first chapter.  You should put chapter/appendix that you
%% write into a separate file, and add a line \include{yourfilename} to
%% main.tex, where `yourfilename.tex' is the name of the chapter/appendix file.
%% You can process specific files by typing their names in at the 
%% \files=
%% prompt when you run the file main.tex through LaTeX.
\chapter{Technology research}

This chapter aims to synthesize the research carried on the main knowledge areas involved in the development of this project. It analyses technologies, paradigms and different insights the design of the project is based upon. This includes the main areas of concern: Public Interface and Interoperability, Database, Real time, Web Technologies, Event logging.   

\section{Public Interface and Interoperability}

\subsection{Client devices}

In recent years, there have been a general steady reduction in costs of hardware devices. Particularly, some microcontrollers have reached prices under 10\$ such as ATmega168, which is the one used by Arduino. Based on a simple microcontroller board and its own development environment, Arduino can be used to develop a vast variety of interactive objects. Its CPU speeds ranging from 8 to 84 Mhz, USB and UART ports, digital and analog I/O, EPROM, SRAM and Flash memories, make it a powerful physical computing device. Although there many other affordable microcontroller platforms, Arduino stands out for its easy-to-use programming environment and cross-platform software. It is specially worth mentioning, however, that it is an open-source physical computing platform. Its software is released under an open source license, while the plans of its modules are published under a Creative Commons license.

On the other hand, the decrease in the price of processors for mobile devices, with excellent multimedia capabilities led to the foundation of the Raspberry Pi Foundation and the public release of the first Raspberry Pi in 2012. Unlike Arduino, it consists of a single-board computer aimed at teaching computer science basics. It is shipped with 700 MHz ARM processors and as any other computer, it comes with GPU, video and audio outputs and SD storage, but only its Model B has 100 Mbits Ethernet connection. Although it supports some Linux kernel-based operating systems like Debian GNU/Linux and Arch Linux ARM, it is recommended to run Raspbian, which is a Debian-based free operating system optimized for the Raspberry Pi hardware. These general purpose features and its credit-card size make it a capable computer which can be used in a wide range of scenarios where can replace regular desktop PCs.

Both devices have different aims and capabilities. Arduino is a easy-to-use lower-level physical computing platform, whereas Raspberry Pi beats general purpose PCs in terms of cost and size. Nonetheless, it is not unusual to combine their features attaching them together as a single device, which \cite (Arduberry) and \cite attempts. While Arduino brings I/O capabilities Raspberry Pi lacks, the latter provides computing power.

These features have contributed to their popularity among people involved in technology as well as computing aficionados. They have attracted great interest in the Internet of Things (IoT) community and have had direct impact on its recent growth. Some project are heavily inspired by Arduino extendability such as \cite (Telefonica's Thinking Things), whilst others build their products based on customized Arduino boards. 

This is the case of Smart Citizen \cite a whole platform aimed at generating participatory processes of people in cities thus, creating more effective and optimized relationships between services, technology and communities in the urban environment. The core of the platform is the called Smart Citizen Kit, a hardware device shipped with air, temperature, light, sound and humidity sensors plus a Wi-Fi module to serve as ambient sensor. They started up with Arduino shields to develop a prototype, until eventually coming up with their own specific-purpose arduino-compatible data-processing board.

\section{Database}

Ever growing data set, very small data units and mainly write ops.

\subsection{Relational Databases and NoSQL}

NoSQL
MongoDB most mature
pros:
- scalabiltiy (auto sharding capabilities)
- performance
- administration ease
- several new options, not very production ready, stability

cons: 
- migration from relational design 
- non-ACID
- eventual consistency (although, non achievable in a distributed system)

\section{Real Time}
for frontend redirect to web technologies

\subsection{Asynchronous Messaging Queue}
RabbitMQ
ZeroMQ

impact on client. Async app server. Reactor pattern.

\subsection{Tuple Space}
redis

\section{Web Technologies}

Since the Sir Tim Berners-Lee's first draft of the World Wide Web back in 1989 and his first proposal for the HyperText Markup Language (HTML) \cite{HTMLtags} in 1991 the WWW has experienced a tremendous evolution. Since then, HTML has gone through many revisions. The World Wide Web Consortium (W3C) published many iterations of the standard until the specification HTML 4.01 in 1999. It was not until 2004, when the Web Hypertext Application Technology Working Group (WHATWG) was founded by individuals of Apple, the Mozilla Foundation, and Opera Software, that HTML was proposed to be extended so as to allow the creation of web applications. WHATWG published the First Public Working Draft of the HTML5 specification in 2008. Although parts of HTML5 have been implemented in browsers it hadn't been until 2012 that W3C designated HTML5 as a Candidate Recommendation and it is planned to be released as a stable Recommendation by the end of 2014.


Ajax

HTML5	
	- what is it
	- who
	- future
	- APIs: Canvas, SVG, ... particularly the ones involved in Redch

Browser vendors
	- dramatical improvement of JS engines
	- battle as new explorer engine
	- prefixed CSS properties and non-standarized prefixed APIs

	- Firefox OS, Node.js (JS in the backend), Win8 JS apps, 
	- rich applications GDocs, Chrome book
	- Towards application platform
Canvas, SVG

JS Libraries
	- to solve cross-browser issues/compatibility, old browsers. jQuery
	- D3, widely used. Standard for data visualization. As being the 5th most starred repository in github, more than rails or angular.js, shows.
	- Chart.js

	JS apps -> Redch web application 
	problem of JS apps -> old browsers. Autoupdate
	- Backbone, underscore -> Angular.js, Ember
	- full ecosystem of MVC components, templates systems, etc.

\subsection{Server Sent Events}
specification over standard HTTP. RFC

\subsection{Web Sockets}
buzzword, trend, full duplex
Too often misunderstood, lower level protocol than HTTP, whole new protocol

\section{Event Logging}
Need to get insight of the system functioning. Misbehaviour. Analysis tool.