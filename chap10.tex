%% This is an example first chapter.  You should put chapter/appendix that you
%% write into a separate file, and add a line \include{yourfilename} to
%% main.tex, where `yourfilename.tex' is the name of the chapter/appendix file.
%% You can process specific files by typing their names in at the 
%% \files=
%% prompt when you run the file main.tex through LaTeX.
\chapter{Conclusions}

\section{Conclusions}

- Two iterations of each component. Better way?
- Decisions unsure
- Pay a lot for puppet and infrastructure, trying to much towards one path without stepping back. It might help not to get these benefit, get it working and then try to address them.
- Goals of the project achieved
- DevOps knowledge has much impact and cost.

\section{Further Work}

- Try other options for the unsure
- Investigate PaaS further. Git-based deployments. They do not require DevOps knowledge.


The myriad of open-source software and modern web technologies used in this MS Thesis have provided to be a viable solution for building IT infrastructure for public research centers. Specifically, these technologies have been combined together to build a distributed as a proof-of-concept for a larger initiative, which aims to provide a valuable insight into the actual production of renewable energies at a small scale in Catalonia.

The introduction presents the motivations behind this initiative of the CREAF and outlines the main goals of this project.

Next, a thorough analysis defines the boundaries of this thesis by providing its scope and requirements. This is detailed further in Chapter 3 with a formal specification of the use cases and the whole conceptual model around the measurement and processing of observations.

Then, Chapter 4 details the findings of the deep research process that had been carried out to assess the best technologies available and later support the design decisions taken in Chapter 5. This is particularly relevant due to the fact the chosen technologies are the basis for its further development.

Chapters 6 and 7 provide insight into the implementation and the infrastructure the system runs on. Particular attention is given to the automation of common processes such as provisioning, deployment and maintenance tasks, which provide reliability and confidence to the system's managers.
