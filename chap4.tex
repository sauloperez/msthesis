%% This is an example first chapter.  You should put chapter/appendix that you
%% write into a separate file, and add a line \include{yourfilename} to
%% main.tex, where `yourfilename.tex' is the name of the chapter/appendix file.
%% You can process specific files by typing their names in at the 
%% \files=
%% prompt when you run the file main.tex through LaTeX.
\chapter{Specification}

The following section describes what the system does detailing its features.

\section{Uses Case Model}

This section describes the services of the systems as events triggered by external actors and their interrelation.

\subsection*{Actors}

The actors of the system are:

\begin{description}
	\item[Sensor device] The device responsible for registering itself in the system and sending the gathered observations to it. As the design and building of the sensor device is out of the scope of the project, it is simulated by a CLI tool that allows to issue request against the system.
	\item[Web application user] A person who interacts with the public web application that shows the observations in a data visualization.
\end{description}

\subsection{Use Cases}

\begin{figure}[ht]
	\centering
	\includegraphics[width=\textwidth]{uses_cases}
	\caption{System's use cases}
	\label{fig:use_cases}
\end{figure}

\begin{usecase}
	\addtitle{Use Case 1}{Insert Sensor}
	\addfield{Actors:}{Sensor device}
	\addfield{Preconditions:}{The system is running}
	\addfield{Postconditions:}{The sensor is registered and persisted in the system}
	\addscenario{Main Success Scenario:}{
		\item The sensor issues a POST request to the sensors endpoint
		\item The system stores the sensor's information in the database
		\item The system sends an HTTP response with 201 status code to the sensor
	}
	\addscenario{Extensions:}{
		\item[1.a] The sensor is already registered
			\begin{enumerate}
			\item[1.] The system returns an HTTP response with status 400
			\end{enumerate}
	}
\end{usecase}

\begin{usecase}
	\addtitle{Use Case 2}{Insert Observation}
	\addfield{Actors:}{Sensor device}
	\additemizedfield{Preconditions:}{
		\item The system is running
		\item The sensor is registered in the system	
	}
	\additemizedfield{Postconditions:}{
		\item The observation is persisted in the system
		\item The observation is sent to the web application tier
	}
	\addscenario{Main Success Scenario:}{
		\item The sensor issues a POST request to the observations endpoint
		\item The system stores the observation's data in the database
		\item The system sends the observation's data to the web application tier
		\item The system sends a HTTP response with 201 status code to the sensor
	}
	\addscenario{Extensions:}{
		\item[1.a] The sensor is not found
			\begin{enumerate}
			\item[1.] The system returns a 400 HTTP response status
			\end{enumerate}
	}
\end{usecase}

\begin{usecase}
	\addtitle{Use Case 3}{Browse data}
	\addfield{Actors:}{Web application user}
	\addfield{Preconditions:}{The system is running}
	\addfield{Postconditions:}{The web application is shown}
	\addscenario{Main Success Scenario:}{
		\item The user points the browser to the web application URL
		\item Once loaded, the web application establishes a stream connection
		\item The system pushes the observations to the web application as they are available
	}
\end{usecase}

\section{Conceptual Model}

As shown in figure \ref{fig:use_cases}, the system consists of two main components the Observations service and the Web application, which operate over the various entities involved in the system. The diagram \ref{fig:conceptual_model} describes them and their relations, heavily based on the OGC O\&M model \cite{OM}.

An observation is an aggregation of the following six elements:

\begin{description}
\item[Feature of interest] A representation of a real-world object that carries the observed property, e.g. "Pant\`a de Sau". Hence, for an in-place instrument it would be the sensor location, whereas for a remote sensor it is the target location.
\item[Procedure] Instance of a process which has performed the observation. Even though it is usually a physical sensor, it can also be a process that leads to an observation such as a computation or the result of a post-processing.
\item[Observed property] Represent the phenomena under observation. Usually a concept of an ontology, e.g. air temperature.
\item[Phenomenon time] Time when the observation's result applies.
\item[Result time] Time when the observation's result has been created. Note these two times may be identical.
\item[Result] Result of the observation, which can be either a scalar value or a complex multi-dimensional array.
\end{description}

\begin{figure}[ht]
	\centering
	\includegraphics[width=\textwidth]{conceptual_model}
	\caption{Conceptual model}
	\label{fig:conceptual_model}
\end{figure}
\newpage

\section{Sequence Diagrams}

\begin{figure}[ht]
	\centering
	\includegraphics[scale=.75]{insert_sensor_seq}
	\caption{Sequence Diagram - Insert Sensor}
	\label{fig:insert_sensor_seq}
\end{figure}

\begin{figure}[ht]
	\centering
	\includegraphics[scale=.75]{insert_observation_seq}
	\caption{Sequence Diagram - Insert Observation}
	\label{fig:insert_observation_seq}
\end{figure}

\begin{figure}[ht]
	\centering
	\includegraphics[scale=.75]{browse_data_seq}
	\caption{Sequence Diagram - Browse Data}
	\label{fig:browse_data_seq}
\end{figure}
