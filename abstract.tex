% $Log: abstract.tex,v $
% Revision 1.1  93/05/14  14:56:25  starflt
% Initial revision
% 
% Revision 1.1  90/05/04  10:41:01  lwvanels
% Initial revision
% 
%
%% The text of your abstract and nothing else (other than comments) goes here.
%% It will be single-spaced and the rest of the text that is supposed to go on
%% the abstract page will be generated by the abstractpage environment.  This
%% file should be \input (not \include 'd) from cover.tex.

Recent development in web technology and infrastructure services together with enhancements in microcontrollers and hardware devices enable the implementation of cheaper IT systems. This enable research centers to build powerful and affordable infrastructures that can ease their work. This is particularly interesting for The Center for Ecological Research and Forestry Applications (CREAF) which is geared towards the creation of new methodological tools in the field of the terrestrial ecology. The upcoming idea of the sensor web -- led by the Open Geospatial Consortium (OGC) -- offers a new way to obtain data on a more interoperable basis.

The aim of this thesis is to implement a first prototype of a larger project whose goal is to provide a system that enables monitoring the distribution of renewable energy produced in Catalan homes in real-time. A thorough research evaluates the available technologies and lays the foundation of the further development of the project. Through an asynchronous messaging queue the system provides a loosely coupled architecture that enables its scalability. A simple single-page web application offers a real-time data visualization of the data generated by sensor simulators which allow the evaluation of the system while the physical sensor devices are not implemented yet.
