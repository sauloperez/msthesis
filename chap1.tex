
%% This is an example first chapter.  You should put chapter/appendix that you
%% write into a separate file, and add a line \include{yourfilename} to
%% main.tex, where `yourfilename.tex' is the name of the chapter/appendix file.
%% You can process specific files by typing their names in at the 
%% \files=
%% prompt when you run the file main.tex through LaTeX.
\chapter{Introduction}

The Center for Ecological Research and Forestry Applications (CREAF) is a public research institution that was created in 1987.  The members of the Governing Council of CREAF are the Generalitat of Catalonia, the Autonomous University of Barcelona (UAB), University of Barcelona (UB), the Institute for Research and Technology (IRTA), the Institute of Catalan Studies (IEC) and the Spanish National Research Council (CSIC).

Its objective is to generate knowledge and create new methodological tools in the field of terrestrial ecology in order to improve environmental planning and management in rural and urban areas with special emphasis on forest ecology. This is achieved, among other means, through the development of methodological and conceptual tools designed to facilitate decision-making and improve environmental management.

Since its creation, CREAF has made very important contributions to the field of terrestrial ecology and towards a sustainable management of the environment. This has been achieved through research, development, training and technology transfer. Some of its most relevant contributions include the design and implementation of the Ecological and Forest Inventory of Catalonia (EFIC), innovative at the international level due to the incorporation of new ecological parameters, the production of the Land Cover Map of Catalonia (MCSC), a high-resolution digital map for environmental assessment and territorial planning and management and the development of the MiraMon \copyright  Geographic Information System, widely adopted in Catalan administration and currently used in over thirty countries around the world.


\section{Motivations}

The motivation for this master's thesis stems from the idea that the wide development of web technologies, DevOps and infrastructure services make possible the implementation of powerful systems with significant cuts on budget and maintenance costs. Moreover, these technologies often entail considerable improvements in maintainability, performance and reduced complexity.

I am firmly convinced that these technologies and tools can improve research in centres such as CREAF, providing them with better and affordable infrastructures reducing the timespan of common processes. Furthermore, they provide new means for the dissemination of the resulting data.

Nonetheless, I think that computer engineering should be conceived as a tool to push forward the development of other sciences. As recently graduated engineers we can contribute back to society with our knowledge in an attempt to solve problems that benefit us all. Hence, I wish to develop a project in which the outcome could improve the work of  public research centres thus making it a useful tool.

Given the interest aroused in topics like distributed computing, sensor networks and resilience systems during in my recent stay in University of Antwerp, I am eager to expand my knowledge further and apply them in a real-world use case.

Within the context of volunteered geographic information (VGI) and renewable energies, CREAF wants to solve the problem of ascertaining the distribution of renewable energy produced in Catalan homes. Nowadays, its performance, time evolution and distribution is unknown thus complicating the decision-making process regarding renewable energy sources in Catalonia.

On the other hand, CREAF wishes to expand its methodological tools by adopting sensor web. The reduced cost of hardware devices like Raspberry Pi and Arduino  and their general-purpose features make them an affordable and versatile solution as sensor devices. Their considerable computational power also facilitates the development of service clients in widely adopted languages. For these reasons, CREAF plans to deploy its own sensor devices in natural surroundings in the near future. 

Although CREAF already took some design decisions, the architecture of the system and the software the devices would be shipped with were still to be determined.

Given the mutual interest in the project outlined by CREAF, we set out to design and implement a prototype as a first working solution.

\section{Project Goals} \label{project_goals}

The Renewable Energy Production Distribution Map of Catalan Homes (REDCH) is aimed at developing a system that offers features to registered users who freely share their data as well as other publicly available features. It will visualize the energy production of the client’s system and its contribution to the whole Catalan renewable energy production in a real time map, while offering a private analytics dashboard to registered users where they can figure out the actual performance of their system.

Given the extent of the desired product with this thesis we wish to develop a proof of concept -- a distributed computing system that will provide essential features, being a simple but functional prototype of the final product. Once built, the system results and metrics will be evaluated and its architecture may eventually become the standard infrastructure basis for future CREAF projects that demand sensor data. As a consequence, all features for registered users are out of the scope.

To sum up, these two general objectives are translated into the following specific goals:

\begin{itemize}
	\item Provide a command-line interface which allows for the simulation of sensor functionality
	\item Develop a functional system that stores and processes sensor observations
	\item Implement a simple public web application to display the observations in a real time map
\end{itemize}

\section{Methodology}

\subsection{Iterative Development}

Although advocating agile software methodologies, the concept-proof nature of the thesis -- which is developed by just one person -- make them an unsuitable choice. We opt instead for a custom adaptation of iterative development.

In conjunction with incremental development, Iterative development is a way of breaking down the development of a software system into smaller chunks and repeated cycles. In each cycle, known as iteration, the slice of functionality is designed, developed, tested, deployed and evaluated. This allows software developers to apply the knowledge acquired in previous iterations, so the first implementation whose goal is to build a bare minimal functional system is iteratively enhanced so as to meet the requirements.

Nevertheless, iterative and incremental development are the basis for Agile Development. Therefore, by adhering to these two practices, we attempt to avoid the agile practices and constraints that may be pointless in this case. Doing so, we aim to progressively enhance the codebase in subsequent iterations, gain insight into the architecture and improve any weak points we may identify until eventually meeting the requirements. Additionally, early results can be achieved and evaluated by CREAF resulting in a smoother collaboration.

\subsection{Test-Driven Development}

The chosen methodology also includes Test-Driven Development, which is a developer practice that involves writing tests before writing the code to be tested. The initially failed test defines the behaviour of the code to be written, then the developer writes the minimum amount of code required to pass the test. Once it passes, it is time to refactor it to remove any duplication. This cycle must be repeated as many times as required to further extend the responsibilities of the code.

Besides validating the correctness of the code, by running the design through test cases the developer is mainly concerned with the interface of the program rather than its actual implementation.

What we aim for by using TDD in this project is to obtain a more modularized, maintainable, and extensible code. The development of the software in small units leads to smaller, more focused and loosely coupled classes and cleaner interfaces. The main benefit we may get by this means, however, is a greater level of confidence in the code caused by the fact that all code written is covered by at least one test.

Additionally, in this early stage of the project to have a test-covered code is basic practice for the successful evolution of the project. So it can be ensured that the intended behaviour is kept and any defects are caught early in the development process and therefore has a considerably less impact on costs than in later stages.
