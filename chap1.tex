%% This is an example first chapter.  You should put chapter/appendix that you
%% write into a separate file, and add a line \include{yourfilename} to
%% main.tex, where `yourfilename.tex' is the name of the chapter/appendix file.
%% You can process specific files by typing their names in at the 
%% \files=
%% prompt when you run the file main.tex through LaTeX.
\chapter{Introduction}

The Center for Ecological Research and Forestry Applications (CREAF) is a public research institution that was created in 1987.  The members of the Governing Council of CREAF are the Generalitat of Catalonia, the Autonomous University of Barcelona (UAB), University of Barcelona (UB), the Institute for Research and Technology (IRTA), the Institute of Catalan Studies (IEC) and the Spanish National Research Council (CSIC).

Its objective is to generate knowledge and create new methodological tools in the field of terrestrial ecology, with special emphasis on forest ecology, in order to improve environmental planning and management in rural and urban areas. This is achieved, among other means, through the development of methodological and conceptual tools designed to facilitate decision-making and improve environmental management.

Since its creation, CREAF has made very important contributions to the field of terrestrial ecology and towards a sustainable management of the environment. This has been achieved through research, development, training and technology transfer. Some of its outstanding breakthroughs are the design and implementation of the Ecological and Forest Inventory of Catalonia (EFIC), innovative at the international level due to the incorporation of new ecological parameters, the production of the Land Cover Map of Catalonia (MCSC), a high-resolution digital map for environmental assessment and territorial planning and management and the development of the MiraMon \copyright Geographic Information System, widely adopted in Catalan administration and currently being used in over thirty countries around the world.


\section{Motivations}

The motivation for this master thesis arise from the idea that the wide development of the web technology such as HTML5, PaaS, IaaS, NoSQL and the steady increase in Open Source adoption in recent years make possible the implementation of powerful systems with significant cuts on budget. Moreover these technologies often come with (entail?) considerable improvements in maintainability, performance and complexity reduction.

In particular, I was convinced that this set of technologies and tools could improve research in centres such as CREAF, providing them with better and affordable infrastructures that allow the timespan reduction of common processes and calculations as well as make possible to dissemination of the valuable data that results from these processes.

I tend to think that computer engineering should be considered as a tool to push forward the development of other sciences. In particular, as recently graduated engineers we can contribute with our acquired knowledge to the society solving problems that benefit us all. Therefore, I wished to develop a project whose outcome could benefit the work of some public research centre becoming a useful tool.

Given my discovered interest in topics like distributed computing, sensor networks and resilience systems arouse in my recent stay in University of Antwerp I eager to expand my knowledge farther and dig deeper into these fields relating them in a real-world use case.

\subsection{CREAF future tools}

On the other hand, CREAF wishes to expand its methodological tools by adopting sensor web. The reduced cost of hardware devices like Raspberry Pi and Arduino  and their general-purpose features make them an affordable option as sensor devices and facilitates the development of sensor client software. For these reasons, CREAF aims to deploy its own sensor devices in wild nature in the near future. 

Although, some points were already clear, some others such as the architecture of the whole system to which the sensor clients will connect to as well as the software they will be shipped with were unknown for CREAF.

Additionally, within in the context of volunteered geographic information (VGI) and renewable energies, wants to solve the problem of knowing the distribution of renewable energy produced in Catalan homes, whose impact is that the actual distribution of the energy produced by either wind or solar systems and their performance and evolution over time are unknown. 

