%% This is an example first chapter.  You should put chapter/appendix that you
%% write into a separate file, and add a line \include{yourfilename} to
%% main.tex, where `yourfilename.tex' is the name of the chapter/appendix file.
%% You can process specific files by typing their names in at the 
%% \files=
%% prompt when you run the file main.tex through LaTeX.
\chapter{Introduction}

The Center for Ecological Research and Forestry Applications (CREAF) is a public research institution that was created in 1987.  The members of the Governing Council of CREAF are the Generalitat of Catalonia, the Autonomous University of Barcelona (UAB), University of Barcelona (UB), the Institute for Research and Technology (IRTA), the Institute of Catalan Studies (IEC) and the Spanish National Research Council (CSIC).

Its objective is to generate knowledge and create new methodological tools in the field of terrestrial ecology, with special emphasis on forest ecology, in order to improve environmental planning and management in rural and urban areas. This is achieved, among other means, through the development of methodological and conceptual tools designed to facilitate decision-making and improve environmental management.

Since its creation, CREAF has made very important contributions to the field of terrestrial ecology and towards a sustainable management of the environment. This has been achieved through research, development, training and technology transfer. Some of its outstanding breakthroughs are the design and implementation of the Ecological and Forest Inventory of Catalonia (EFIC), innovative at the international level due to the incorporation of new ecological parameters, the production of the Land Cover Map of Catalonia (MCSC), a high-resolution digital map for environmental assessment and territorial planning and management and the development of the MiraMon \copyright  Geographic Information System, widely adopted in Catalan administration and currently being used in over thirty countries around the world.


\section{Motivations}

The motivation for this master thesis arise from the idea that the wide development of web technology such as HTML5, PaaS, IaaS, NoSQL and the steady increase in Open Source adoption in recent years make possible the implementation of powerful systems with significant cuts on budget. Moreover these technologies often come with (entail?) considerable improvements in maintainability, performance and reduced complexity.

I am firmly convinced that this set of technologies and tools could improve research in centres such as CREAF, providing them with better and affordable infrastructures that allow to reduce the timespan of common processes and calculations. Moreover, they provide new means for the dissemination of the valuable data that result from these processes.

I think that computer engineering should be conceived as a tool to push forward the development of other sciences. In particular, as recently graduated engineers we can contribute with our acquired knowledge to the society attempting to solve problems that benefit us all. Therefore, I wish to develop a project whose outcome could improve the work of some public research centre becoming a useful tool.

Given my discovered interest in topics like distributed computing, sensor networks and resilience systems arouse in my recent stay in University of Antwerp I am eager for expanding my knowledge farther and dig deeper into these fields relating them in a real-world use case.

Within in the context of volunteered geographic information (VGI) and renewable energies CREAF wants to solve the problem of knowing the distribution of renewable energy produced in Catalan homes. Nowadays the actual distribution of the energy produced by either wind or solar systems and their performance and evolution over time are unknown complicating the decision-making process regarding renewable energies in Catalonia.

On the other hand, CREAF wishes to expand its methodological tools by adopting sensor web. The reduced cost of hardware devices like Raspberry Pi and Arduino  and their general-purpose features make them an affordable option as sensor devices and facilitates the development of sensor client software. For these reasons, CREAF aims to deploy its own sensor devices in wild nature in the near future. 

Although some points were already clear, some others such as the architecture of the whole system to which the sensor clients will connect to as well as the software they would be shipped with were unknown for CREAF.

\section{Project Goals}

This project is aimed to develop a system that offers features to registered users’, those who freely share their data, and some other features available for all internet users. It will visualize the energy production of the clients’ system and its contribution to the whole Catalan renewable energy production in a real time map, while offering a private analytics dashboard to registered users where they can figure out the actual performance of their system.

Given the extent of the desired product we will develop a proof of concept; a distributed computing based system containing base features being a simple yet functional prototype of the final product. Once built the system results and metrics will be evaluated and its architecture may eventually become the standard infrastructure basis for future CREAF projects that demand sensor data.

Hence, the private features for registered users are out of the scope. TODO extend to clearly specify what is going to be out of the scope

TODO cite that the sensor devices are going to be either Raspberry Pi or Arduino

Hence, these two general goals are translated to the following specific goals (describe further the fact that we have a reduced scope; we don't have time for all):

\begin{itemize}
	\item Provide a command line interface that allows the simulation of the sensors functionality
	\item Develop a functional system that stores and processes the sensor observations
	\item Implement a simple public web application to show the real time data
\end{itemize}

\section{Methodolgy}

Although advocating for agile software methodologies the concept-proof nature of the project and the fact that it's currently developed by just one person make them an unsuitable choice. We vote for a custom adaptation of iterative development, instead.

Iterative development, in conjunction with incremental development, is a way of breaking down the development of a software system in smaller chunks and repeated cycles. In each cycle, referred to as iteration, the slice of functionality is designed, developed, tested, deployed and evaluated. This allows software developers to apply the knowledge acquired in previous iterations, so the first implementation whose goal is to build a bare minimal functional system is iteratively enhanced so as to meet the requirements.

Nevertheless, iterative and incremental development are the basis for Agile development. Therefore, by adhering to these basis we attempt to avoid the agile practices and constraints that may be pointless in this case. Doing so, we aim to progressively enhance the codebase in subsequent iterations, gain insight into the architecture and improve any weak points we may identify until eventually meeting the requirements. Additionaly, in this way early results can be achieved and evaluated by CREAF resulting in a smoother collaboration.

TDD benefits. TDD related to iterative methodology (it is also based in iterations). Why suits iterative method.
