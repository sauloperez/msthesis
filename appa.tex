\chapter{Instruction Manual}

\section{Web Application}\label{doc-webapp}

\subsection*{Installation}\label{installation}

First, clone the repo:

\begin{verbatim}
git clone git@github.com:sauloperez/redch-webapp.git
\end{verbatim}

Next, install its dependencies:

\begin{verbatim}
bundle install
\end{verbatim}

REDCH Webapp gets observations from a RabbitMQ, so make sure the
RabbitMQ server is running and accessible from within your network. For
Mac OS X users this is done be typing:

\begin{verbatim}
rabbitmq-server
\end{verbatim}

While for Ubuntu users this is done with:

\begin{verbatim}
sudo /etc/init.d/rabbitmq-server start
\end{verbatim}

Besides, you must load the appropriate Procfile containing the values
for the required env variables. It must contain the following:

\begin{verbatim}
AMQP_HOST=<RabbitMQ_server_host>
\end{verbatim}

Name this file after the environment, e.g. \texttt{development} and save
it wherever you like from your directory tree. You can edit the
development file in
\url{https://github.com/sauloperez/redch-webapp/blob/master/development}.

You can find further documentation in
\href{https://devcenter.heroku.com/articles/procfile\#developing-locally-with-foreman}{Process
Types and the Procfile} from Heroku Dev Center and from its
\href{https://github.com/ddollar/foreman}{Github repo}.

\subsection*{Usage}\label{usage}

\subsubsection*{Development}\label{development}

Now we are ready to start the webapp. From the root folder type the
following in your terminal, using the path of your environment file:

\begin{verbatim}
foreman start -e <path_to_env_file>
\end{verbatim}

Nevertheless, it is recommended to have a development environment file
per machine ignored by git, so any customizations can be made for that
machine.

That's all. The webapp is up and running. Point your browser to
\texttt{http://localhost:3000} and you will see the real time map.

\subsubsection*{Production}\label{production}

In production the deployment process is automatized using
\href{http://capistranorb.com/}{Capistrano}. To deploy just type the
following command from your machine:

\begin{verbatim}
cap production deploy
\end{verbatim}

This essentially runs commands on the remote server through SSH. Once
the process is finished, point your browser to the production server.

Additionaly, some tasks to manage production services are provided. Each
service has its own start, stop and restart actions:

\begin{verbatim}
cap production nginx:restart
cap production passenger:stop
cap production rabbitmq:start

# Or open an SSH connection
cap production utils:ssh
\end{verbatim}

To list all available tasks type:

\begin{verbatim}
cap production -T
\end{verbatim}

\subsection*{Testing}\label{testing}

Testing covers both frontend and backend of the app. Jasmine has been
chosen for the former while RSpec for the latter.

To test the frontend start up the server as stated above and point your
browser to \texttt{http://localhost:3000/SpecRunner.html}. You will get
immediate results of how many test are passing (hopefully all of them).

As for the backend, type the following in your terminal:

\begin{verbatim}
rspec spec
\end{verbatim}

This will execute all tests contained in the \texttt{/spec} folder.


\section{CLI Client}\label{doc-cli-client}

\subsection*{Installation}\label{installation}

You must clone the repo

\begin{verbatim}
$ git clone git@github.com:sauloperez/redch.git
\end{verbatim}

Although not mandatory, it is highly recommended to add the executable
in your PATH environment variable. To do so, create a softlink in a
suitable system folder pointing to \texttt{/bin/redch} of the previously
cloned repository. In Mac OS X \texttt{/usr/local/bin} might be a good
choice. You can do that with the following command:

\begin{verbatim}
$ ln -s ~/redch/bin/redch /usr/local/bin/.
\end{verbatim}

Then, make sure your PATH variable looks up into the folder that
contains the softlink. If not, add it. Doing so \texttt{redch} will be
globally accessible.

In case of working with Bash shell this should be set in the
\texttt{\textasciitilde{}/.bash\_profile} file. Find further details in
\href{http://www.joshstaiger.org/archives/2005/07/bash_profile_vs.html}{.bash\_profile
vs .bashrc}

\begin{verbatim}
# Prepend the variable with the right path
PATH="/usr/local/bin:/usr/local/sbin:$PATH"
\end{verbatim}

Lastly, load the changes

\begin{verbatim}
$ source ~/.bash_profile
\end{verbatim}

\subsection*{Usage}\label{usage}

The command-line interface comes with the methods \texttt{setup} and
\texttt{simulate} that you can use as follows.

\subsubsection*{Setup}\label{setup}

If no coordinates are provided, the \texttt{setup} command will pick up
a random location near by Tàrrega within a range of 90 Km as the sensor
location. It uses the first MAC address of the system as device unique
identifier.

\begin{verbatim}
$ redch setup -c '41.65, 2.13'
\end{verbatim}

\subsubsection*{Simulate}\label{simulate}

The \texttt{simulate} command loads the configuration set by the
\texttt{setup} and issues randomly generated observations for each time
period indefinitely. If not specified a period of 2 seconds will be
used.

\begin{verbatim}
# Issue a post request each second
$ redch simulate -p 1
\end{verbatim}

If the setup defaults suit you, you can skip the \texttt{setup} and just
type the \texttt{simulate} command. It always executes the setup before
the simulation if no configuration is found.

\begin{verbatim}
# Set up the sensor with its defaults and simulate observations
$ redch simulate
\end{verbatim}

\subsubsection*{Help}\label{help}

You can always list the available commands with the \texttt{help}
command or the \texttt{-h} flag

\begin{verbatim}
$ redch help
Commands:
  redch help [COMMAND]  # Describe available commands or a single one
  redch setup     # Sets up the environment to enable the use of ...
  redch simulate  # Simulate a sensor generating electrical power ...
\end{verbatim}

Or find out the details of a particular command

\begin{verbatim}
$ redch help setup
Usage:
  redch setup

Options:
  c, [--coordinates=COORDINATES]

Sets up the environment to enable the use of the device
\end{verbatim}


\section{Sensor Observation Service}\label{doc-sos}

\subsection*{Installation}\label{installation}

\subsubsection*{Database}\label{database}

You must create a Postgres PostGIS database named \texttt{sos}. To do
so, connect to postgres and create the database. Then, connect to it and
add the PostGIS extension.

\begin{verbatim}
$ psql -h localhost

=# CREATE DATABASE sos;
=# \connect sos;
=# CREATE EXTENSION postgis;

# Quit from the DB
=# \q
\end{verbatim}

\subsubsection*{AMQP Service extension}\label{amqp-service-extension}

This SOS implementation has been extended to suit the requirements of
the REDCH project. A RabbitMQ client wrapper has been added as
extension.

Contained in the
\href{https://github.com/sauloperez/sos/tree/master/src/extensions/amqp-service}{amqp-service}
submodule of the \texttt{extensions} module, it comes with an example
properties file which can be found in the
\href{https://github.com/sauloperez/sos/tree/master/config}{config
folder}. These properties are the following:

\begin{verbatim}
# URL of the RabbitMQ server
redch.amqp.host=192.168.0.20

# Name of the exchange to where the observations should be published
redch.amqp.exchange=observations
\end{verbatim}

An updated copy of this file must be stored in the same folder named as \\
\texttt{redch.properties}.

\textbf{Note:} Make sure you compile the \texttt{amqp-service} submodule
whenever you change it before compiling the whole SOS. You can do so
with the \texttt{mvn install} command.

\paragraph{Integration}\label{integration}

This service has been integrated with the SOS by using a subclass of the
corresponding request handler. As every observation received must be
published into the queue, the
\href{https://github.com/sauloperez/sos/blob/master/src/bindings/rest/code/src/main/java/org/n52/sos/binding/rest/resources/observations/ObservationsPostRequestHandler.java}{ObservationsPostRequestHandler}
with
\href{https://github.com/sauloperez/sos/blob/mastero/src/bindings/rest/code/src/main/java/org/n52/sos/binding/rest/resources/observations/RedchObservationsPostRequestHandler.java}{RedchObservationsPostRequestHandler}
of the rest binding, which connects to the AMQP Service.

\subsection*{Usage}\label{usage}

\subsubsection*{Development}\label{development}

First of all, deploy the Java Webapp with the command
\texttt{mvn clean tomcat:deploy} or
\texttt{mvn clean tomcat:undeploy tomcat:deploy} if it already exists.
Then configure it using the webapp. Just browse to
\texttt{\textless{}tomcat\_url\textgreater{}/webapp} and follow the
wizard's steps.

REDCH SOS talks to RabbitMQ and Postgresql. So besides setting them up
and running, make sure the \texttt{redch.amqp.host} property of the
\texttt{redch.properties} file points to the right the RabbitMQ server.
As for the DB, make sure the field \texttt{Host} under
\textbf{Datasource settings} section of the SOS administrative backend
points to the right host.

\subsubsection*{Production}\label{production}

In production the deployment process is automatized using
\href{http://capistranorb.com/}{Capistrano}. To deploy just type the
following command from your machine:

\begin{verbatim}
cap production deploy
\end{verbatim}

This essentially runs commands on the remote server through SSH. Once
the process is finished, point your browser to
\texttt{\textless{}production\_server\textgreater{}/webapp} and fill up
the wizard fields like in development.

In addition, it also provides tasks to manage Tomcat. You can list all
available tasks typing:

\begin{verbatim}
$ cap production -T
...
cap tomcat:compile                 # Compile tomcat
cap tomcat:deploy                  # Deploy tomcat
cap tomcat:restart                 # Restart tomcat
cap tomcat:start                   # Start tomcat
cap tomcat:stop                    # Stop tomcat
cap tomcat:undeploy                # Undeploy tomcat
\end{verbatim}

So, you can run any of these by tying, for instance:

\begin{verbatim}
cap production tomcat: restart
\end{verbatim}

\subsection*{Testing}\label{testing}

The AMQPService extensions as well as the SOS itself come with unit
tests made with JUnit. It is recommended to run them from your IDE of
choice. Most of them have JUnit plugins available.

The tests can be found in the \texttt{src/test} folder of every module.

\clearpage
\newpage
